\documentclass[11pt]{beamer}
\definecolor{links}{HTML}{2A1B81}
\hypersetup{colorlinks,linkcolor=,urlcolor=links}
\usetheme{Frankfurt}
\setbeamersize{text margin left=13pt, text margin right=15pt}
\definecolor{mygreen}{cmyk}{0.5,0,0.5,0.5} % for red
% https://tex.stackexchange.com/questions/66465/how-to-get-actual-values-of-colour-theme-colours-in-beamer
%\definecolor{mygreen}{cmyk}{0.82,0.11,1,0.25} % for green
\definecolor{myblue}{RGB}{33,84,157}
\setbeamertemplate{blocks}[rounded][shadow=false]
\addtobeamertemplate{block begin}{\pgfsetfillopacity{0.8}}{\pgfsetfillopacity{1}}
\setbeamercolor{structure}{fg=blue!50}
\setbeamercolor*{block title example}{fg=blue!50,
bg= blue!10}
\setbeamercolor*{block body example}{fg= blue,
bg= blue!5}

\usepackage[utf8]{inputenc}
\usepackage[english]{babel}
\usepackage{amsmath}
\usepackage{listings}
\usepackage{fancyhdr,lastpage}
\usepackage{url}
\usepackage{hyperref}
 \usepackage{multirow}
\usepackage{amsfonts}
\usepackage{xcolor}
\usepackage{amssymb}
\usepackage{caption}
%\captionsetup{font=scriptsize,labelfont=scriptsize}
\usepackage{graphicx}
\usepackage{multicol}
\usepackage{tikz}
\usepackage{lastpage}
\usepackage{amsmath}
\usepackage{layout}
\usepackage{fontawesome}
\usepackage{marvosym}
\newcommand\Fontvi{\fontsize{6}{6}\selectfont}
\newcommand\Fontvii{\fontsize{11}{11}\selectfont}
\newcommand\Fontviii{\fontsize{9}{9}\selectfont}
\newcommand\Fontviiii{\fontsize{3}{3}\selectfont}
\DeclareMathOperator{\E}{\mathbb{E}}
\newcommand\norm[1]{\left\lVert#1\right\rVert}
\newcommand\abs[1]{\left\lvert#1\right\rvert}
\newcommand{\vecX}{\mathbf{X}}
\newcommand{\vecZ}{\mathbf{Z}}
\newcommand{\vecU}{\mathbf{U}}
\newcommand{\vecI}{\mathbf{I}}
\newcommand{\vecO}{\mathbf{O}}
\newcommand{\vecE}{\mathbf{E}}
\newcommand{\vecM}{\mathbf{M}}
\newcommand{\vecS}{\mathbf{S}}
\newcommand{\vecT}{\mathbf{T}}
\newcommand{\vecx}{\mathbf{x}}
\newcommand{\vecu}{\mathbf{u}}
\newcommand{\vecz}{\mathbf{z}}
\newcommand{\veczero}{\mathbf{0}}
\newcommand{\vecmu}{\mbox{\boldmath$\mu$}}
\newcommand{\veczeta}{\mbox{\boldmath$\zeta$}}
\newcommand{\vecalpha}{\mbox{\boldmath$\alpha$}}
\newcommand{\vecbeta}{\mbox{\boldmath$\beta$}}
\newcommand{\vecgamma}{\mbox{\boldmath$\gamma$}}
\newcommand{\veclambda}{\mbox{\boldmath$\lambda$}}
\newcommand{\vecLambda}{\mbox{\boldmath$\Lambda$}}
\newcommand{\vecomega}{\mbox{\boldmath$\omega$}}
\newcommand{\vecpi}{\mbox{\boldmath$\pi$}}
\newcommand{\matsig}{\mbox{\boldmath$\Sigma$}}
\newcommand{\matTheta}{\mbox{\boldmath$\Theta$}}
\newcommand{\matepsilon}{\mbox{\boldmath$\epsilon$}}
\newcommand{\matPsi}{\mbox{\boldmath$\Psi$}}
\newcommand{\varthet}{\mbox{\boldmath$\vartheta$}}
\newcommand{\ttr}{\text{tr}}
\newcommand{\iprod}{\prod_{i=1}^n}
\newcommand{\isum}{\sum_{i=1}^n}
\newcommand{\gsum}{\sum_{g=1}^G}
\newcommand{\inv}{^{\raisebox{.2ex}{$\scriptscriptstyle-1$}}}
\DeclareMathOperator{\E}{\mathbb{E}}
\DeclareMathOperator{\R}{\mathbb{R}}
\fontsize{10pt}{8.2}\selectfont

\addtobeamertemplate{navigation symbols}{}{%
    \usebeamerfont{footline}%
    \usebeamercolor[fg=black]{footline}%
    \hspace{1em}%
    \insertframenumber/\inserttotalframenumber
}

\newcommand{\docYear}{\text{2022}} % year of assignment due date 
\newcommand{\aUthor}{\text{Anjali Silva}}
\newcommand{\dayMonth}{\text{15 Sept }}
\newcommand{\eMail}{\text{a.silva@utoronto.ca}}
\newcommand{\courseCodeName}{\text{BCB410H1: Applied Bioinformatics }}
\newcommand{\anyQuestions}{\text{Any questions?}}
\newcommand{\imagePath}{\text{/Users/as/Desktop/Images/}}
\title[\courseCodeName \docYear]{\courseCodeName \docYear}
\author[\aUthor]{}


%\setbeamercovered{transparent} 
%\setbeamertemplate{navigation symbols}{} 
%\logo{} 
\date{\vspace{0.5cm} \\ \\
Lecture 1 \\
\dayMonth \docYear \\ \aUthor, PhD\\ \vspace{0.7cm} \Fontviii{\faEnvelope \hspace{0.01cm}  \href{mailto:\eMail}{\eMail} \hspace{0.1cm} \faGithub \hspace{0.01cm} \href{https://anjalisilva.github.io/}{anjalisilva.github.io} \hspace{0.1cm} \faTwitter \hspace{0.01cm} \href{https://twitter.com/silva_anjali?lang=en}{\MVAt Silva\_Anjali}}} 
%\subject{} 

\makeatother
\setbeamertemplate{footline}
{
  \leavevmode%
  \hbox{%
  \begin{beamercolorbox}[wd=.4\paperwidth,ht=2.25ex,dp=1ex,center]{author in head/foot}%
    \usebeamerfont{author in head/foot}\insertshortauthor
  \end{beamercolorbox}%
  \begin{beamercolorbox}[wd=.6\paperwidth,ht=2.25ex,dp=1ex,center]{title in head/foot}%
    \usebeamerfont{title in head/foot}\insertshorttitle\hspace*{3em}
  \end{beamercolorbox}}%
  \vskip0pt%
}
\makeatletter

%\setbeamertemplate{footline}[text line]{%
%  \parbox{\linewidth}{\vspace*{-8pt}Clustering of high-throughput sequencing data\hfill\insertshortauthor\hfill}}
%\setbeamertemplate{navigation symbols}{}

\begin{document}
\definecolor{light-gray}{gray}{0.75}
\section{Intro}
\subsection{Welcome}
\begin{frame}
\titlepage
\end{frame}

%\begin{frame}
%\tableofcontents
%\end{frame}

\subsection{Welcome}
\begin{frame}
\frametitle{Welcome}
\begin{itemize}
\item Instructor: \aUthor, PhD
\vspace{0.2in}
\begin{itemize}
\item Researcher, University of Toronto
\vspace{0.1in}
\item Pronouns: she/her
\vspace{0.1in}
\item Email: \eMail
\begin{itemize}
\vspace{0.1in}
\item \textbf{Use subject line ``BCB410" for emails}.
\vspace{0.1in}
\item E.g., BCB410: inquiry regarding assessment I. 
\end{itemize}
\end{itemize}
\end{itemize}
\end{frame}


\subsection{Outline}
\begin{frame}
\frametitle{Outline}
\begin{itemize}
\item Introduction 
\vspace{0.1in}
\item Syllabus 
\vspace{0.1in}
\item Academic Integrity
\vspace{0.1in}
\item Bioinformatics
\vspace{0.1in}
\item {\sf R}, Packages and RStudio
\vspace{0.1in}
\item Practical
\end{itemize}
\end{frame}



\subsection{BCB410H}
\begin{frame}
\frametitle{BCB410H1}
\begin{itemize}
\item \courseCodeName
\vspace{0.2in}
\begin{itemize}
\item Practical introduction to concepts, standards and tools for the implementation of strategies in bioinformatics and computational biology. \textbf{Student led discussions} plus a strong component of \textbf{hands-on exercises} [\docYear \hspace{0.1mm} UToronto Calendar].
\end{itemize}
\end{itemize}
\end{frame}

\subsection{Prerequisites}
\begin{frame}
\frametitle{Prerequisites}
\begin{itemize}
\item Prerequisites
\vspace{0.2in}
\begin{itemize}
\item BCH311H1/​ MGY311Y1; (CSC324H1/​ CSC373H1/​ CSC375H1) or permission of the course coordinator.
\vspace{0.2in}
\item Will assume prior knowledge of biological systems. 
\vspace{0.2in}
\item Will assume prior knowledge of programming principles.
\vspace{0.2in}
\item Not an introductory course to {\sf R}; if you have no/limited prior knowledge, will have to pick up fast.
\end{itemize}
\end{itemize}
\end{frame}


\subsection{Class}
\begin{frame}
\frametitle{Welcome}
\begin{itemize}
\item Class
\vspace{0.2in}
\begin{itemize}
\item Wednesday 10:00 am – 12 noon EST; online - synchronous.
%\item Classes start at 10 minutes past the hour.
\end{itemize}
\end{itemize}
\end{frame}


\subsection{Aim}
\begin{frame}
\frametitle{Welcome}
\begin{itemize}
\item Aim
\vspace{0.2in}
\begin{itemize}
\item In this year's course you will define a useful tool for the analysis of biological data, write an R package to support it, review and critique other packages, and improve and document your work.
\end{itemize}
\end{itemize}
\end{frame}

\subsection{Phases}
\begin{frame}
\frametitle{Welcome}
\begin{itemize}
\item Content - 5 phases:
\vspace{0.2in}
\begin{itemize}
\item Section I:  Learn {\sf R} and basic structure of  an {\sf R} package. 
\vspace{0.1in}
\item Section II: Define a tool for data analysis.
\vspace{0.1in}
\item Section III: Develop the tool into an  {\sf R} package and keep track using Git. 
\vspace{0.1in}
\item Section IV: Review peer packages.
\vspace{0.1in}
\item Section V: Improve own package based on peer review and final submission.
\end{itemize}
\end{itemize}
\end{frame}

\subsection{Calendar}
\begin{frame}
\frametitle{Course Overview}
\begin{itemize}
\item Tentative Calendar* \\
\vspace{0.2in}
* This may be modified as needed. \\
\vspace{0.2in}
\begin{itemize}
\item See syllabus.
\end{itemize}
\end{itemize}
\end{frame}



\subsection{Assessments}
\begin{frame}
\frametitle{Course Overview}
\Fontvii
\begin{itemize}
\item Assessments
\end{itemize}
\begin{table}[]
\begin{tabular}{ll} 
\hline 
Activity                                                                                                              & Weight                                                            \\ \hline
1) Participation                                                                                                      & see syllabus                                                   
\vspace{0.1in}
\\
2) Journal Entries/Assignments                                                                         & 
\vspace{0.1in}
\\
3) Initial submission of {\sf R} package + Presentation                                                                                        & \begin{tabular}[c]{@{}l@{}} \\  \end{tabular} 
\vspace{0.1in}
\\
\begin{tabular}[c]{@{}l@{}}4) Review of peer {\sf R} packages$^{*}$ \end{tabular} &                                                         
\vspace{0.1in}
\\
\vspace{0.1in}
5) Final submission of R package  &                                                           \\ \hline
Total                                                                                                                 & 100            
\\ \hline                                                 
\end{tabular}
\end{table}
* Number of participation panels maybe altered. But total marks remain the same. 
\end{frame}


\subsection{Assessments}
\begin{frame}
\frametitle{Course Overview}
\begin{itemize}
\item Read syllabus carefully for deadlines and late policy. 
\vspace{0.1in}
\item If any questions, send me an email or see me in office hours. 
\end{itemize}
\end{frame}

\subsection{Lectures}
\begin{frame}
\frametitle{Course Overview}
\begin{itemize}
\item Lecture material will be available on Quercus.
\vspace{0.1in}
\item Assessment submissions will be done on Quercus and GitHub, unless stated otherwise.
\vspace{0.1in}
\item You are responsible for uploading files on time, in the format specified. 
\vspace{0.1in}
\item If technical difficulties with Quercus, must email instructor a copy of submission \underline{before} the deadline.
\vspace{0.1in}
\item Quercus history will be checked and emailed copy will be graded, accordingly. 
\end{itemize}
\end{frame}

%\subsection{Material}
%\begin{frame}
%\frametitle{Course Overview}
%\begin{itemize}
%\item Lecture material will be available on Quercus.
%\vspace{0.1in}
%\item For code, see \url{https://github.com/anjalisilva/BCB410H}.
%\end{itemize}
%\end{frame}



\subsection{Textbooks}
\begin{frame}
\frametitle{Course Overview}
\begin{itemize}
\item Textbooks
\vspace{0.2in}
\begin{itemize}
\item {\textit{R packages} by Hadley Wickham and Jennifer Bryan} \\ \url{https://r-pkgs.org/index.html}
\vspace{0.2in}
\item Available free online.
\end{itemize}
\end{itemize}
\end{frame}


\subsection{Textbooks}
\begin{frame}
\frametitle{Course Overview}
\begin{itemize}
\item Other recommendations:
\vspace{0.2in}
\begin{itemize}
\item \textit{R packages} covers the basic components. 
\vspace{0.2in}
\item More in-depth details at: \\
\vspace{0.1in}
Writing {\sf R} Extensions: Creating R packages \\
\url{https://cran.r-project.org/doc/manuals/R-exts.html#Creating-R-packages}. 
\vspace{0.2in}
\item Recommended once you master the basics. 
\end{itemize}
\end{itemize}
\end{frame}


\subsection{Textbooks}
\begin{frame}
\frametitle{Course Overview}
\begin{itemize}
\item Other recommendations:
\begin{itemize}
\vspace{0.2in}
\item \textit{Advanced R} by Hadley Wickham \\
\url{https://adv-r.hadley.nz/functions.html}
\vspace{0.2in}
\item \textit{What They Forgot to Teach You About R} by Jennifer Bryan and Jim Hester \\
\url{https://rstats.wtf/}
\vspace{0.2in}
\item Available free online.
\end{itemize}
\end{itemize}
\end{frame}


\subsection{Academic Integrity}
\begin{frame}
\frametitle{Course Overview}
\begin{itemize}
\item Academic Integrity
\vspace{0.2in}
\begin{itemize}
\item You are responsible for understanding policies on academic integrity.
\vspace{0.2in}
\item \url{https://www.academicintegrity.utoronto.ca/}
\vspace{0.2in}
\item \url{https://guides.library.utoronto.ca/plagiarism}
\vspace{0.2in}
\item \url{http://steipe.biochemistry.utoronto.ca/abc/index.php/ABC-Plagiarism}
\end{itemize}
\end{itemize}
\end{frame}

\subsection{Academic Integrity}
\begin{frame}
\frametitle{Course Overview}
\begin{itemize}
\item Full disclosure policy for this course:
\vspace{0.1in}
\begin{itemize}
\item If it's not your own, new idea, it has a source.
\vspace{0.1in}
\item All sources must be referenced.
\end{itemize}
\vspace{0.2in}
\item For advice:
\vspace{0.1in}
\begin{itemize}
\item How not to plagiarize \url{https://advice.writing.utoronto.ca/using-sources/}
\end{itemize}
\end{itemize}
\end{frame}


\subsection{Expectations}
\begin{frame}
\frametitle{Course Expectations}
\begin{itemize}
\item When submitting assessment files to Quercus, label  using this format: LASTNAME\_FirstInitial\_\textbf{Assessment}.format.
\vspace{0.1in}
\begin{itemize}
\item E.g., SILVA\_A\_A1.PDF
\vspace{0.1in}
\end{itemize}
\item Instructions of each assessment will specify \textbf{Assessment} name.
\vspace{0.1in}
\item Must follow this format to avoid confusion with name and assessment weight. 
\end{itemize}
\end{frame}


\subsection{Expectations}
\begin{frame}
\frametitle{Course Expectations}
\begin{itemize}
\item Bioinformatics is a highly interdisciplinary field. Be open to different views. 
\vspace{0.2in}
\item Will need Internet access to use  or download free bioinformatics software and tools required for class work. 
\vspace{0.2in}
%\item If you are unable to submit major assignments by deadline, need to provide \underline{1 week} prior notice via email. 
\vspace{0.2in}
%\item Practical assignments will be due 11.59pm EST on the day of the practical. 
\end{itemize}
\end{frame}

\subsection{Expectations}
\begin{frame}
\frametitle{Course Expectations}
\begin{itemize}
\item \textbf{Use subject line ``BCB410" for emails}.
\begin{itemize}
\item E.g. BCB410: inquiry regarding assignment I. 
\end{itemize}
\vspace{0.2in}
\item Weekday: 48h and Weekends: 48h - 72h reply.
\vspace{0.2in}
\item Use practical time to ask questions.
\end{itemize}
\end{frame}



\subsection{Etiquette}
\begin{frame}
\frametitle{Course Etiquette}
\begin{itemize}
\vspace{0.2in}
\item Respectful listening and sharing; One speaker at a time.
\vspace{0.2in}
\item Keep `mute' status unless you need to ask a question. 
\vspace{0.2in}
\item Say your name before asking/answering a question. 
\vspace{0.2in}
\item Personal grade matters must be discussed privately. 
\end{itemize}
\end{frame}




\subsection{Etiquette}
\begin{frame}
\frametitle{Discussion Boards}
\begin{itemize}
\vspace{0.2in}
\item Introduce yourself in the `Get to know each other: BCB410' Discussion Board.
\vspace{0.2in}
\item Use `Lecture/Practical Questions' Discussion Board to ask questions.
\vspace{0.2in}
\item Will also be used to post presentation links for the entire class. 
\vspace{0.2in}
\item Personal grade matters must be discussed privately. 
\end{itemize}
\end{frame}



\subsection{Etiquette}
\begin{frame}
\frametitle{Discussion Boards}
\begin{itemize}
\vspace{0.2in}
\item Respect Discussion Board rules at all times. 
\vspace{0.2in}
\begin{center}
\begin{figure}
\includegraphics[scale=0.35]{/Users/as/Desktop/Images/Figure82.png}
\end{figure}
\end{center} 
\end{itemize}
\end{frame}



\title[\courseCodeName \docYear]{\anyQuestions}
\author[\aUthor]{}
\date{\vspace{0.5cm} \\ \\
\\
\\ \vspace{0.5cm} } 
%\subject{} 


\begin{frame}
\titlepage
\end{frame}


\title[\courseCodeName\docYear]{Let's begin the lecture...}
\author[\aUthor]{}
\date{\vspace{0.5cm} \\ \\
\\
\\ \vspace{0.5cm} } 
%\subject{} 

\section{Bioinformatics}
\begin{frame}
\titlepage
\end{frame}

\subsection{Central Dogma}
\begin{frame}
\frametitle{Central Dogma}
\begin{center}
\begin{figure}
\includegraphics[scale=0.3]{/Users/as/Desktop/Images/Figure4.png}
\caption{The classic view of the central dogma of biology states that genetic information in DNA is transcribed into messenger RNA (mRNA) and each mRNA contain information to synthesize a protein. But with new discoveries, there are exceptions to this. For example, DNA that does not encode proteins may encode different types of functional RNAs. [Image by user EEPUCKETT, 2015; Transcriptomics for Conservation, wildlifesnpits.wordpress.com]}
\end{figure}
\end{center} 
% https://wildlifesnpits.wordpress.com/2015/11/18/transcriptomics-for-conservation/
\end{frame}


\subsection{OMICS}
\begin{frame}
\frametitle{OMICS}
\begin{center}
OMICS is generating large amounts of data that require analysis.  \\
\begin{figure}
\includegraphics[scale=0.15]{/Users/as/Desktop/Images/Figure2.png}
\caption{Central role of bioinformatics in the modern biological investigation based on omics sciences [Facchiano, 2016]. }
\end{figure}
\end{center} 
% https://www.researchgate.net/figure/The-scheme-indicates-the-central-role-of-bioinformatics-in-the-modern-biological_fig1_313544643
\end{frame}


\subsection{Bioinformatics}
\begin{frame}
\frametitle{Bioinformatics}
\begin{itemize}
\item Term ``bioinformatics" was coined by \textbf{Paulien Hogeweg} and \textbf{Ben Hesper} in \textbf{1970s} to describe ``the study of informatic processes in biotic systems" [Hogeweg, P. \textit{PLoS Computational Biology}, 2011].
\vspace{0.2in}
\item Can broadly be defined as the science of storing, retrieving and analysing large amounts of biological information [ebi.ac.uk, 2018].
%https://www.ebi.ac.uk/training/online/course/bioinformatics-terrified-2018/what-bioinformatics
\vspace{0.2in}
\item  Involves information related to biological macromolecules such as \textbf{DNA, RNA, proteins and metabolites}. 
\end{itemize}
\end{frame}

\subsection{Bioinformatics}
\begin{frame}
\frametitle{Bioinformatics}
\begin{center}
\begin{figure}
\includegraphics[scale=0.3]{/Users/as/Desktop/Images/Figure1.png}
\caption{Bioinformatics is a highly interdisciplinary, fast growing field. This figure from De Maio et al., 2018 shows the number of publications in PubMed on Bioinformatics.}
\end{figure}
\end{center} 
\end{frame}

\subsection{Tools}
\begin{frame}
\frametitle{Tools}
\begin{center}
\begin{figure}
\includegraphics[scale=0.23]{/Users/as/Desktop/Images/Figure10.png}
\caption{Many bioinformatics tools and resources are available on the internet. However, biological datasets can be very large, requiring command-line tools to manipulate the data. [Pevsner, Bioinformatics And Functional Genomics, 3rd ed].}
% Image from Bioinformatics and Functional Genomics, 3rd Edition, by Jonathan Pevsner
\end{figure}
\end{center} 
\end{frame}

\title[\courseCodeName\docYear]{Databases...}
\author[\aUthor]{}
\date{\vspace{0.5cm} \\ \\
\\
\\ \vspace{0.5cm} } 
%\subject{} 

\subsection{Resources}
\begin{frame}
\titlepage
\end{frame}

\subsection{NCBI}
\begin{frame}
\frametitle{Central Bioinformatics Resources}
\begin{center}
\begin{figure}
\includegraphics[scale=0.3]{/Users/as/Desktop/Images/Figure11.png}
\vspace{0.2in}
\caption{NCBI is one of the central bioinformatics sites. NCBI develops/ maintains databases, software for searching and analysis of data. Link: \url{https://www.ncbi.nlm.nih.gov}.}
\end{figure}
\end{center} 
\end{frame}

\subsection{NCBI}
\begin{frame}
\frametitle{Central Bioinformatics Resources}
\begin{center}
\begin{figure}
\includegraphics[scale=0.25]{/Users/as/Desktop/Images/Figure12.png}\vspace{0.2in}
\caption{NCBI is one of the central bioinformatics sites.}
\end{figure}
\end{center} 
\end{frame}

\subsection{NCBI}
\begin{frame}
\frametitle{Central Bioinformatics Resources}
\begin{center}
\begin{figure}
\includegraphics[scale=0.45]{/Users/as/Desktop/Images/Figure13.png}
\end{figure}
\end{center} 
\end{frame}

\subsection{INSDC}
\begin{frame}
\frametitle{Examples of Databases}
\begin{center}
\begin{figure}
\includegraphics[scale=0.4]{/Users/as/Desktop/Images/Figure6.png}
\caption{INSDC (\url{http://www.insdc.org/}) coordinates DNA sequence data. GenBank is maintained by National Center for Biotechnology Information (NCBI), European Nucleotide Archive (ENA) is maintained by European Molecular Biology Laboratory - European Bioinformatics Institute (EMBL-EBI). DDBJ is the DNA Data Bank of Japan.}
% Image from Bioinformatics and Functional Genomics, 3rd Edition, by Jonathan Pevsner
\end{figure}
\end{center} 
\end{frame}


\subsection{RNA}
\begin{frame}
\frametitle{Examples of Databases}
\begin{center}
\begin{figure}
\includegraphics[scale=0.3]{/Users/as/Desktop/Images/Figure7.png}
\caption{Different databses available for RNA (\url{https://rnacentral.org/expert-databases}), including RNA alignments, RNA sequences and RNA structures.}
% https://rnacentral.org/expert-databases
\end{figure}
\end{center} 
\end{frame}

\subsection{UniProt}
\begin{frame}
\frametitle{Examples of Databases}
\begin{center}
\begin{figure}
\includegraphics[scale=0.3]{/Users/as/Desktop/Images/Figure8.png}
\caption{UniProt: the \textbf{Universal Protein} provides high-quality and freely accessible resource of protein sequence and functional information. The UniProt Consortium comprises of the EBI, the SIB and the PIR. Prior to 2002, the Swiss-Prot and TrEMBL were protein databases maintained by EBI and SIB. Protein Sequence Database (PSD) was a protein database maintained by PIR. Around 2002, these three databases were merged to form the UniProt.}
% https://www.slideshare.net/kew15na/the-uni-prot-knowledgebase
\end{figure}
\end{center} 
\end{frame}

\title[\courseCodeName\docYear]{\anyQuestions}
\author[\aUthor]{}
\date{\vspace{0.5cm} \\ \\
\\
\\ \vspace{0.5cm} } 
%\subject{} 


\begin{frame}
\titlepage
\end{frame}


\title[\courseCodeName\docYear]{{\sf R}}
\author[\aUthor]{}
\date{\vspace{0.5cm} \\ \\
\\
\\ \vspace{0.5cm} } 
%\subject{} 
\section{R}
\subsection{R}
\begin{frame}
\titlepage
\end{frame}

\subsection{R}
\begin{frame}
\frametitle{What is {\sf R}?}
\begin{itemize}
\item A language and environment for statistical computing and graphics.
\vspace{0.2in}
\item {\sf R} was initially written by Ross Ihaka and Robert Gentleman.
\vspace{0.2in}
\item Since mid-1997, the R Core Team modify the {\sf R} source.
\vspace{0.2in}
\item {\sf R} runs on a wide variety of UNIX platforms, Windows and MacOS. 
% taken from https://cran.r-project.org/doc/FAQ/R-FAQ.html#Why-is-R-named-R_003f
\end{itemize}
\end{frame}


\subsection{R}
\begin{frame}
\frametitle{{\sf R} continue...}
\begin{itemize}
\item {\sf R} is a scripting language, thus an interpreter executes commands one line at a time.
\vspace{0.15in}
\item A Free software under the terms of the GNU General Public License.
\vspace{0.15in}
\item {\sf R} home page: \url{https://www.R-project.org/}
\vspace{0.2in}
\item How can {\sf R} be obtained? 
\begin{itemize}
\item Via CRAN, the ``Comprehensive R Archive Network".
\item \url{https://cran.r-project.org/}
\end{itemize}
\end{itemize}
% taken from https://cran.r-project.org/doc/FAQ/R-FAQ.html#Why-is-R-named-R_003f
\end{frame}


\subsection{R}
\begin{frame}
\frametitle{{\sf R} continue...}
\begin{itemize}
\item How can  {\sf R} be installed?
\vspace{0.1in}
\begin{itemize}
\item Unix
\item  \url{https://cran.r-project.org/doc/FAQ/R-FAQ.html\#How-can-R-be-installed-\_0028Unix\_002dlike\_0029}
\vspace{0.2in}
\item Windows 
\item \url{https://cran.r-project.org/bin/windows/base/}
\vspace{0.2in}
\item Mac 
\item \url{https://cran.r-project.org/bin/macosx/}
\end{itemize}
\end{itemize}
% taken from https://cran.r-project.org/doc/FAQ/R-FAQ.html#Why-is-R-named-R_003f
\end{frame}


\subsection{R}
\begin{frame}
\frametitle{{\sf R} continue...}
\begin{center}
\begin{columns}[t]
        \column{.5\textwidth}
        \includegraphics[width=\columnwidth,height=5cm]{/Users/as/Desktop/Images/Figure15.png}
        \column{.4\textwidth}
        \includegraphics[width=\columnwidth,height=4.5cm]{/Users/as/Desktop/Images/Figure16.png}
    \end{columns} 
\end{center}
\end{frame}



\subsection{R}
\begin{frame}
\frametitle{{\sf R} continue...}
\begin{itemize}
\item {\sf R} can be used interactively or non-interactively.
\vspace{0.2in}
\item Interactively, with or without an integrated development environment (IDE): RStudio.
\vspace{0.2in}
\item Non-interactively via scripts.
\vspace{0.2in}
\item {\sf R} is designed with interactive data exploration in mind.
\vspace{0.2in}
\item A version of {\sf R} is released each year. Current release is 4.0.2.
\end{itemize}
\end{frame}



\subsection{R packages}
\begin{frame}[fragile]
\frametitle{Documentation for {\sf R}}
\begin{itemize}
\item Online documentation for functions and variables in {\sf R} exists.
\vspace{0.2in}
\item Obtained by typing \textit{help(FunctionName)} or  \textit{?FunctionName} at the {\sf R} prompt, where FunctionName is name of function.
\vspace{0.2in}
\item E.g., if `sum' is the function then: 
\vspace{0.1in}
\begin{semiverbatim}
> help(sum)
> ?sum
\end{semiverbatim}
% taken from https://cran.r-project.org/doc/FAQ/R-FAQ.html#Why-is-R-named-R_003f
\end{itemize}
\end{frame}

\subsection{R packages}
\begin{frame}
\frametitle{{\sf R} packages}
\begin{itemize}
\item Mechanism for extending the basic functionality of {\sf R}.
\vspace{0.2in}
\item It is natural to put together many functions together into a package achieving a specific goal.
\begin{itemize}
\item Function for preprocessing data. 
\item Function for clustering data.
\item Function for selecting best cluster.
\item Function to visualize the clustering results.
\item Put together = Package for Clustering. 
\end{itemize}
\vspace{0.2in}
\item Provide a defined interface, with inputs (arguments) and outputs (return values).
\end{itemize}
\end{frame}

\subsection{R packages}
\begin{frame}
\frametitle{{\sf R} packages}
\begin{itemize}
\item Building R packages requires tools that must be in place before process of development can start.
\vspace{0.2in}
\item Mainly \href{https://cran.r-project.org/}{R} and \href{https://www.rstudio.com/products/rstudio/download/}{RStudio} (recommended).
\vspace{0.2in}
\item Mac OS
\begin{itemize}
\item Xcode development environment
\item \url{https://apps.apple.com/us/app/xcode/id497799835?mt=12}
\end{itemize} 
\vspace{0.2in}
\item Windows
\begin{itemize}
\item Rtools 
\item \url{https://cran.r-project.org/bin/windows/Rtools/}
\end{itemize} 
\end{itemize}
\end{frame}


\subsection{R packages}
\begin{frame}[fragile]
\frametitle{{\sf R} packages: Mac OS}
\begin{itemize}
\item For more information: \url{https://r-pkgs.org/setup.html}
\vspace{0.2in}
\item Mac OS
\begin{itemize}
\item Xcode development environment
\item \url{https://apps.apple.com/us/app/xcode/id497799835?mt=12}
\end{itemize} 
\vspace{0.2in}
\item Then, in the shell, do:
\begin{itemize}
\begin{verbatim}
xcode-select --install
\end{verbatim}
\end{itemize} 
\end{itemize}
\end{frame}


\subsection{R packages}
\begin{frame}
\frametitle{{\sf R} packages: Windows}
\begin{itemize}
\item Windows:
\begin{itemize}
\item Rtools 
\item \url{https://cran.r-project.org/bin/windows/Rtools/}
\vspace{0.2in}
\end{itemize} 
\item For more information: \url{https://r-pkgs.org/setup.html}
\vspace{0.2in}
\item During the Rtools installation you may see a window asking you to “Select Additional Tasks”.
\vspace{0.1in}
\begin{itemize}
\item Do not select the box for ``Edit the system PATH". devtools and RStudio should put Rtools on the PATH automatically when it is needed.
\item Do select the box for ``Save version information to registry". It should be selected by default.
\end{itemize}
\end{itemize}
\end{frame}


\subsection{R packages}
\begin{frame}
\frametitle{{\sf R} packages: Linux}
\begin{itemize}
\item For more information: \url{https://r-pkgs.org/setup.html}
\vspace{0.2in}
\item Install {\sf R}, but also the {\sf R}  development tools. For example, on Ubuntu (and Debian) you need to install the r-base-dev package.
\end{itemize}
\end{frame}

\subsection{R packages}
\begin{frame}
\frametitle{What {\sf R} packages are available?}
\begin{itemize}
\item \href{https://cran.r-project.org/web/packages/}{CRAN}
\begin{itemize}
\item $>$16K packages [as of \docYear] 
\item \url{https://cran.r-project.org/web/packages/}
\end{itemize}
\vspace{0.2in}
\item \href{https://bioconductor.org/packages/release/bioc/}{Bioconductor}
\begin{itemize}
\item $>$1900 packages [as of \docYear] 
\item \url{{https://bioconductor.org/packages/release/bioc/}}
\end{itemize}
\vspace{0.2in}
\item \href{https://github.com/search?q=r+packages&type=Repositories}{GitHub}
\begin{itemize}
\item $>$ 63K results [as of \docYear] 
\item \url{https://github.com/search?q=r+packages&type=Repositories}
\end{itemize}
\end{itemize}
\end{frame}

\subsection{RStudio}
\begin{frame}
\frametitle{{\sf R}Studio}
\begin{itemize}
\item \href{https://www.rstudio.com/products/rstudio/download/}{RStudio} is not required to build {\sf R} packages.
\vspace{0.2in}
\item However, it contains many features that make the development process easier and faster. 
%\vspace{0.2in}
%\item To setup the environment prior to package development use this \href{Building R Packages Pre-Flight Check List}{check list}.
\end{itemize}
\end{frame}

\subsection{RStudio}
\begin{frame}
\frametitle{{\sf R}Studio}
\begin{center}
\begin{figure}
\includegraphics[scale=0.27]{/Users/as/Desktop/Images/Figure17.png}
\caption{Anatomy of RStudio. 1. This is the Console. 2. Environment and History. 3. Files, Plots, Packages, Help and Viewer. }
% https://dzchilds.github.io
\end{figure}
\end{center} 
\end{frame}


\subsection{RStudio}
\begin{frame}
\frametitle{{\sf R}Studio}
\begin{center}
\begin{figure}
\includegraphics[scale=0.27]{/Users/as/Desktop/Images/Figure83.png}
\caption{Tools $\rightarrow$ Keyboard Shortcuts Help. }
% https://dzchilds.github.io
\end{figure}
\end{center} 
\end{frame}

\title[\courseCodeName\docYear]{\anyQuestions}
\author[\aUthor]{}
\date{\vspace{0.5cm} \\ \\
\\
\\ \vspace{0.5cm} } 
%\subject{} 

\begin{frame}
\titlepage
\end{frame}



\title[\courseCodeName\docYear]{Practical}
\author[\aUthor]{}
\date{\vspace{0.5cm} \\ \\
\\
\\ \vspace{0.5cm} } 
%\subject{} 

\section{Practical}
\subsection{Practical}
\begin{frame}
\titlepage
\end{frame}


\subsection{Practical}
\begin{frame}
\frametitle{{\sf R}Studio}
\begin{itemize}
\item Let's open up RStudio. 
\vspace{0.1in}
\begin{center}
\begin{figure}
\includegraphics[scale=0.18]{/Users/as/Desktop/Images/Figure17.png}
\end{figure}
\end{center} 
\item On Console, get working directory:
\begin{semiverbatim}
> getwd()
\end{semiverbatim}
\vspace{0.1in}
\item {To set to desired directory \\
Session $\rightarrow$ Set Working Directory  $\rightarrow$  Choose Directory...}
\end{itemize}
\end{frame}

\subsection{Practical}
\begin{frame}
\frametitle{{\sf R}Studio}
\begin{itemize}
\item Alternatively, you may use: 
\begin{semiverbatim}
> setwd("/../../..") 
\end{semiverbatim}
\vspace{0.1in}
\item {To open a new script: \\
File $\rightarrow$ New File $\rightarrow$  R Script}
\vspace{0.1in}
\item {Save this: \\
File $\rightarrow$ Save $\rightarrow$ Practical\_StudentName.R}
\vspace{0.1in}
\item Practical\_StudentName.R is called a script. 
\end{itemize}
\end{frame}

\subsection{Practical}
\begin{frame}[fragile]
\frametitle{{\sf R} Features}
\begin{itemize}
\item In {\sf R}, the indexing begins from 1.
\vspace{0.1in}
\item {\sf R} is case sensitive (``X" is not the same as ``x").
\vspace{0.1in}
\item {\sf R} uses dynamic variable typing, so variables can be used over and over again.
\end{itemize}
\end{frame}

\subsection{Practical}
\begin{frame}[fragile]
\frametitle{Assignment and Commenting}
\begin{itemize}
\item The $\leftarrow$ symbol is the assignment operator.
\vspace{0.1in}
\item To assign a value to a variable called `test1'
\begin{verbatim}
test1 <- 123 
test1
\end{verbatim}
\vspace{0.1in}
\item Comment using \# character 
\begin{verbatim}
test1 <- 123  # This is a comment 
test1 # This is called auto-printing
\end{verbatim}
\end{itemize}
\end{frame}


\subsection{Practical}
\begin{frame}[fragile]
\frametitle{Over-writing}
\begin{itemize}
\item From previous slide we had:
\begin{verbatim}
test1 <- 123 
test1
\end{verbatim}
\vspace{0.1in}
\item Over-write previous value of the ’test1’ variable with a new value:
\begin{verbatim}
test1 <- test1 + 2
test1 # 125
\end{verbatim}
\vspace{0.1in}
\item Over-write previous value of the ’test1’ variable with a new value:
\begin{verbatim}
test1 <- 5 + 2
test1 # 7
\end{verbatim}
\end{itemize}
\end{frame}


\subsection{Practical}
\begin{frame}[fragile]
\frametitle{Version}
\begin{itemize}
\item To obtain session information
\begin{verbatim}
sessionInfo()
\end{verbatim}
\vspace{0.1in}
\item Version information:
\begin{verbatim}
R.Version()
\end{verbatim}
\vspace{0.1in}
\item Show objects in workspace
\begin{verbatim}
ls()
\end{verbatim}
\end{itemize}
\end{frame}


\subsection{Practical}
\begin{frame}[fragile]
\frametitle{{\sf R} Built-in Functions}
\begin{itemize}
\item There are many built-in functions. You will learn these as you go. 
\vspace{0.1in}
\item The ``argument" of the function is provided inside the brackets.
\vspace{0.1in}
\item The ``return value" of the function is the value provided back.
\vspace{0.1in}
\item We will cover some basic functions: 
\begin{verbatim}
x <- 5
x # auto-printing
print(x) # explicit printing
class(x) # "numeric"
typeof(x) # "double"
length(x) # 1
\end{verbatim}
\vspace{0.1in}
\end{itemize}
\end{frame}


\subsection{Practical}
\begin{frame}[fragile]
\frametitle{{\sf R} Built-in Functions}
\begin{itemize}
\item Return value from functions can be assigned to a variable or printed:
\begin{verbatim}
x <- 5
x # auto-printing

y <- x + 5 
y # 10

z <- tyepof(y) # return value  assigned to variable
z # "double"
\end{verbatim}
\vspace{0.1in}
\end{itemize}
\end{frame}


\subsection{Practical}
\begin{frame}[fragile]
\frametitle{{\sf R} Help Function}
\begin{itemize}
\item Getting help:
\vspace{0.1in}
\begin{verbatim}
?"<-"  # help on assignment operator
help("<-") # help on assignment operator

?typeof # help on typeof function

?class # help on class function

?print # help on print function
\end{verbatim}
\end{itemize}
\end{frame}

\title[\courseCodeName\docYear]{\anyQuestions}
\author[\aUthor]{}
\date{\vspace{0.5cm} \\ \\
\\
\\ \vspace{0.5cm} } 
%\subject{} 

\begin{frame}
\titlepage
\end{frame}


\subsection{Practical}
\begin{frame}[fragile]
\frametitle{{\sf R} Data Types}
\begin{itemize}
\item Numeric: floating types (double precision).
\vspace{0.1in}
\item Logicals: booleans = TRUE/FALSE or T/F.
\vspace{0.1in}
\item Character strings.
\vspace{0.1in}
\item Examples:
\begin{itemize}
\vspace{0.1in}
\begin{verbatim}
xValue <- 100
xValue
 
yVariable <- FALSE
yVariable
 
zVariable <- "hello"
zVariable
\end{verbatim}
\end{itemize}
\end{itemize}
\end{frame}



\subsection{Practical}
\begin{frame}[fragile]
\frametitle{{\sf R} Class}
\begin{itemize}
\item Numbers in R are usually treated as numeric objects (i.e. double precision real numbers). 
\vspace{0.1in}
\item To explicitly assign an integer, need to specify the L suffix. 
\vspace{0.1in}
\begin{verbatim}
x <- 1L
x
class(x) # "integer"
\end{verbatim}
\end{itemize}
\end{frame}


\subsection{Practical}
\begin{frame}[fragile]
\frametitle{{\sf R} Class}
\begin{itemize}
\item Complex class: 
\vspace{0.1in}
\begin{verbatim}
x <- c(2 + 0i, 5 + 4i)
class(x) # "complex"
\end{verbatim}
\vspace{0.1in}
\item Inf represents infinity:
\begin{verbatim}
Inf
1 / Inf # 0
\end{verbatim}
\vspace{0.1in}
\item NaN represents an undefined value/missing value:
\begin{verbatim}
NaN # not a number
0 / 0 # NaN 
\end{verbatim}
\end{itemize}
\end{frame}



\subsection{Practical}
\begin{frame}[fragile]
\frametitle{Concatenating}
\begin{itemize}
\item c() function concatenating elements together:
\begin{verbatim}
x <- c(0.5, 0.6)
class(x) # "numeric"

x <- c("a", "b", "c")
class(x) # "character"

x <- c(TRUE, FALSE)
class(x) # "logical"
\end{verbatim}
\end{itemize}
\end{frame}


\subsection{Practical}
\begin{frame}[fragile]
\frametitle{Character Strings}
\begin{itemize}
\item Character strings are collections of characters.
\vspace{0.1in}
\item Provided as values in single or double quotes.
\vspace{0.1in}
\begin{verbatim} 
xVariable <- `hello'
class(xVariable) # "character"

zVariable <- "hello"
class(zVariable) # "character"
\end{verbatim}
\vspace{0.1in}
\item ``paste" converts inputs to strings, concatenate and return:
\vspace{0.1in}
\begin{verbatim} 
paste(xVariable)
\end{verbatim}
\end{itemize}
\end{frame}



\subsection{Practical}
\begin{frame}[fragile]
\frametitle{Character Strings}
\begin{itemize}
\item ``cat" concatenates and prints the arguments to the screen:
\vspace{0.1in}
\begin{verbatim} 
cat("\n", xVariable, zVariable) # "\n" adds new line
\end{verbatim}
\vspace{0.1in}
\item ``print" prints the argument:
\begin{verbatim} 
print(c(zVariable, xVariable))
\end{verbatim}
\end{itemize}
\end{frame}





\subsection{Practical}
\begin{frame}[fragile]
\frametitle{Missing Values}
\begin{itemize}
\item Missing values are denoted by NA (Not Available) or NaN (Not a Number).
\vspace{0.1in}
\begin{itemize}
\begin{verbatim} 
x <- c(1, 3, NA, 4, 5)
class(x) # "numeric"

y <- c(1, 3, NaN, 4, 5)
class(y) # "numeric"

# is.na() is used to test objects if they are NA
# is.nan() is used to test for NaN

is.na(x) # FALSE FALSE  TRUE FALSE FALSE
is.nan(x) # FALSE FALSE FALSE FALSE FALSE
\end{verbatim}
\end{itemize}
\end{itemize}
\end{frame}



\title[\courseCodeName\docYear]{Question: What is the difference between NA and NaN in {\sf R}?}
\author[\aUthor]{}
\date{\vspace{0.5cm} \\ \\
\\
\\ \vspace{0.5cm} } 
%\subject{} 


\begin{frame}
\titlepage
\end{frame}



\title[\courseCodeName \docYear]{\anyQuestions}
\author[\aUthor]{}
\date{\vspace{0.5cm} \\ \\
\\
\\ \vspace{0.5cm} } 
%\subject{} 


\begin{frame}
\titlepage
\begin{itemize}
\item To do: Journal Entry 1 (Note, may need a distribution of Latex installed). 
\item Take a look at 'Initial submission + Presentation of R package'.
\end{itemize}
\end{frame}





\subsection{Practical}
\begin{frame}
\frametitle{Practical}
\begin{itemize}
\item Today we looked at the following topics. 
\vspace{0.1in}
\begin{itemize}
\item Assignment and Commenting
\item Over-writing
\item Built-in Functions
\item Help
\item Classes
\item Concatenating
\item Character Strings
\item Missing Values
\end{itemize}
\end{itemize}
\end{frame}

\subsection{Practical}
\begin{frame}
\frametitle{Practical - Tips for Solving Issues}
\begin{itemize}
\item Copy and paste the entire \textbf{exact} error message into Google.
\begin{itemize}
\item Someone else may have gotten this same error and has asked a question.
\end{itemize}
\vspace{0.1in}
\item Copy and paste the entire error message into Google, followed by `r'.
\vspace{0.1in}
\item Google the name of the function with term `tutorial r' to see tutorials.
\vspace{0.1in}
\item If struggling with code for a plot, Google `r plot plotname', then click on Images.
\vspace{0.1in}
\item If errors with reading files, ensure path is correct. Check using getwd().
\end{itemize}
\end{frame}



\end{document}
